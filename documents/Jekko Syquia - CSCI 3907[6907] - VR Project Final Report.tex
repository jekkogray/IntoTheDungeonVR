\documentclass[12pt, twoside]{article}
\usepackage[a4paper,width=150mm,top=25mm,bottom=25mm,bindingoffset=0mm]{geometry}
\usepackage[colorinlistoftodos]{todonotes}
\usepackage{gensymb}
\usepackage{fancyhdr}
\fancyhead{}
\fancyhead[RO,LE]{Into The Dungeon}
\usepackage{setspace}
\usepackage{listings}
\lstset{
  basicstyle=\ttfamily,
  columns=fullflexible,
  frame=single,
  breaklines=true,
  postbreak=\mbox{\textcolor{red}{$\hookrightarrow$}\space},
}
\usepackage{hyperref}
\hypersetup{colorlinks=true, linkcolor=black, filecolor=black,      
    urlcolor=black,}

\pagestyle{fancy}
\doublespace

\title{Into the Dungeon}
\author{Jekko Syquia}

\begin{document}
\maketitle
\titlepage

\tableofcontents
\newpage


\section{Introduction}

\emph{Into the dungeon} is a dungeon monster slayer VR game targeted for the
Oculus Quest device tethered on PC. The objective of the game is to go through
each level and try to slay all the monsters present in each level of the game.
Each level introduces new weapons, and with each new level the user is
introduced to new enemies. New enemies deal harder damage as the game
progresses. Each enemy AI is able to target the user and will shoot projectiles
at the user when the user is at a closer range to the player. Every enemy also
has their own score value that increases the users total score in the game. \par

This game main objective is to allow the users to experience streamlined games
like \emph{Doom} and other first-person shooter games in VR, thus enabling a
tensed gaming experience. Unlike other streamlined VR games however, the user
has a longer health bar than usual to allow the user to play at a more relaxed
phased in the game. If the user dies, the user is respawned at the starting
position of that level and the enemies are respawned. Users can also expect to
keep a score count in the game to show off their friends. It is also important
to notice that this game does not require the user to finish each level as they
have the option of skipping to the next level if they would like. Therefore, it
is a "finish your game at your own phase" and the score is dependent on the play
time of the user.
\newpage

\section{VR Application Design}
One of the most important key design decision when designing this game was
balancing between physical engagement and motion sickness prevention. \par 

Balancing is most notable in the exclusion of weapons that require physical
contact with the enemies i.e. swords, hammers etc. This decision was important
as too much physical engagement such as swinging the sword could disorient users
by having them move and swing at the same time. This can make users
uncomfortable and also would require more space to work around in the game.\par

The grabbing system in the game has also been changed. While the user can still
interact with objects closely, interaction ray has been introduced to allow
users to point at a weapon at which it will be redirected to their hands at any
distance. This design decision prevents the need for users to physically reach
their weapon when it is dropped on the floor.\par

Gameplay wise, users have infinite amount of ammo, this decision was taken into
account to prevent users from needing to reload and prevents added complexity of
loading their weapons which would make this game less arcade feeling.\par

The map design simplicity was also reduced a significant amount. Instead of
moving around the map so much, the level was simplified to prevent users from
getting lost in the game.

\section{Integration}
For the build-process, this project was written entriely in C\# Unity and
Blender was the tool for modeling my enemies and level design. The weapons in
the game was provided through the asset store in Unity.\par 

The tool mainly used for handling VR interaction in the game was the inclusion
of
\emph{\href{https://docs.unity3d.com/Packages/com.unity.xr.interaction.toolkit@0.9/manual/index.html}{XR
Interaction Toolkit}} through the package manager. This tool provided the tools
necessary for handling any VR device. For example, interaction with objects in
the game and player in game setup was handled by this tool. Mechanics such as
hand presence still had to be scripted and written individually, but it provided
a great framework for utilizing the device features such as the haptic feedback
for the weapons. \par

Models in the game was built using Blender and exported as \emph{.fbx} files
into Unity. This allowed for custom map models in the game and designing custom
enemies not found in the asset store. The creation and baking of textures was
also handled through Blender. Once imported in Unity as an asset an enemy script
is attached to handle their behavior.

\section{User Experience}
\emph{Into the Dungeon} has many design considerations to allow for a relaxed
user experience. In this section we talk about the design decisions that make
this game much more enjoyable. It is important to note that the game was meant
to replicate arcade games like \emph{Doom} as mentioned earlier. However the
game is also meant as a way for users to be physically active, but each level is
not required be completed as the user can skip through each level as they would
like.

\subsection{Navigation}
Users can expect to move in the game via teleportation, controller or physical
movement. Since the game utilizes \emph{Six degrees of Freedom (DOF)} users are
not limited to teleportation. Teleportation is introduced in the game to
mitigate the unconformability when moving with a joystick. Furthermore, if the
user has the space like a large room, the users can use their physical space in
the game. While the games utilizes all three, they are not restricted to one and
can choose any option as they wish.\par 

To ease the movement even further, the user has the option for \emph{Snap
Rotation} functionality. Snap Rotation allows users to flick the right analog
stick left or right to a fixed degree of 45\textdegree. This prevents the need
to rotate physically and addresses concerns with wire management when connected
to a computer.\par

An important design decision when it comes to moving from one level to another
was the exclusion of doors. To make the game even easier to move around, users
can navigate to each level using the UI present at beginning every of every
level. This prevents the need to interact with doors to go to the next level of
the game.

\subsection{Guiding Audio/Sound Effects}
Every weapon and enemy interaction is present in the game. For example, the
handgun has handgun sound effects for firing. Additionally, once the shot has
been fired sound effects will be present in the location of the bullet.
Therefore, there is a sense of 3D spatial sound in the game. Once the enemy is
defeated the game also lets the user know the enemy is dead by indicating an
explosion sound depending on the enemy. \par

Enemies in the game have their own individual sound as well. Each sound source
will be relative to the location of the player. By doing this, we can have
spatial awareness in the audio. Therefore, enemies shot far away will have a
more muted sound, enemies shot closer will sound closer. This rule applies to
all elements in the game to further enhance the presence of enemies in the
game.\par 

Levels also has background music that is playing throughout the game.
Furthermore, when a player finishes a level, a sound cue is played to indicate
to the player that level is done and will be moving to the next level.

\subsection{Text Instructional}
In the starting room the user will be given a tutorial on how to move about the
room i.e. how to use the teleportation. Additionally in the beginning room the
user will be provided with instructions on how to use the weapons by
initializing UI elements that indicates what everything is. Whenever the user
enters a new room an instruction on how to use that weapon will appear. Once the
user has read and tested the item the user can then proceed to walk forward to
engage the enemies which they can immediately notice. 

\subsection{Instructional Signs/Wayfinders}
Each room contains an instruction on how to use the weapons as well as how to
move to the next room as the user is spawned in front of the instructions.

\subsection{Six Degrees of Freedom}
\emph{Six Degrees of Freedom (DOF)} is present in the game. This design decision
was an important implementation as moving with guns around the game would be a
lot harder if the user does not have full control of their weapons.

\section{User Guideline}

\subsection{Windows Installation}
To use the game for Windows. Extract the \emph{Windows Build V 1.2.7z} to their
desired location using \textbf{\href{https://www.7-zip.org/download.html}{7zip}}.
Assuming the user has already installed the necessary drivers to connect their
VR to a PC. Simply execute the file \emph{Into The Dungeon VR.exe} located in
the Windows build folder. There are no further installations needed as the
application is built.
\subsection{Android Installation}
To use the game for Android, make sure that the user is in Developer mode and
install the oculus ADB driver on their PC.

\textbf{\href{https://developer.oculus.com/downloads/package/oculus-go-adb-drivers/}{Oculus}}.
Once the user has downlaoded this, using the command prompt or terminal
\emph{cd} to the location of the adb and run the following:j
\begin{lstlisting}{language=Bash}
adb install -r C:\Users\YourName\Downloads\MyCoolNewApp.apk
\end{lstlisting}
On their device, users can then navigate to the app library and go to unknown
sources to play use the application. The game will proceed to placing you in a
tutorial setting where you can view the controls. For best experience use the
teleportation (left trigger)

\section{Conclusion}
Considering the initial design what works is the room to room dynamics of
the game. The physical movement from one room to another was removed to
prevent the user from getting lost. However, I do believe implementing UI
room navigation was a lot more effective. What did not work
was implementing swords and such as since it was a lot more complicated to
implement physical damage through user collisions since I need to take account
for constant collision. movement. What does need work is the
level design, as I focused more on reducing unnecessary movements this also over
simplified the level design in the game.\par

\section{Suggestions for Further Improvements}
To improve the game even more, perhaps the introduction of more complicated map
designs could make the game more engaging and increase the repeatability factor
of the game. Also including more weapons in the game can increase the difficult
as well as being able to carry multiple weapons can be helpful for the user. To
enhance the experience it would be helpful to optimize the models even further
to be compatible with the Oculus Quest unlinked. This application runs smoothly
connected to a PC. The game has no proper ending aside from defeating the big
boss in the game, this decision was made in part with relaxed gameplay with the
game.\par
\newpage
\appendix
\section{Lessons Learned}
In designing this VR application my vision for what I want was pretty clear and
achievable, however some complexities had to be cut down due to time
constraints. The biggest challenge I faced was implementing enemy AI behaviour,
collisions, and user movements. \par 

AI behavior was certainly one of the biggest hurdles trying to implement in the
game. As the enemy does not only move in 2D but have to compensate for 3D ray
casting from the user in different direction. One of the struggles I initially
had was calculating the path way for the enemy to go in the event of an obstacle
in the way. This was however solved using NavMeshAgent that is built in Unity.
This allowed me to implement enemies without having the need to calculate if the
next step is possible as the component calculates the next possible step to the
destination. Within this script I was also able to implement different states
for the enemy with the following: Patrol, Chase, Attack. This was also a
struggle to implement as it was hard to calculate the threshold of when the
patrol, chase and attack occurs. However, I found that 20, 10, and 5 to be the
best spot for attacking.\par

The next struggle in the game was calculating the Collision and when it should
happen. I discovered that the method \emph{OnCollisionEnter(Collision)} and
\emph{OnCollisionStay(Collision)} were two different things. When calculating
enemy damage I used on \emph{OnCollisionEnter(Collision)} as using the alter
would keep decreasing my health when a projectile collided with the player
especially if the projectile had a slower velocity. This was the main reason I
excluded using swords in the game as I did not know when the damage should keep
inflicting or when an enemy actually attack the enemy opposed to having the
sword just touching the enemy. I have also added collision cues by adding
particle effects when the bullet collides with an enemy.\par

User movement was a struggle however, when I found out about \emph{XR
Interaction Toolkit} the implementation was simplified. Once the rig was setup
it was primarily implementing the behavior of weapon objects and applying
scripts to them that correspond to user button press such as the shoot button. 

Beyond the project I have also implented particle effects when the enemy is hit.
This was beyond the requirements of the game, and including AI Enemies that
shoot back was also beyond the requirements. Building the models myself was also
beyond the requirements of the game, as I aimed to make the game more varied.

\section{Feedback to the Instructor}
For the VR Dev platform, it was very helpful to use Unity with C\# as the
learning was transferrable to other game developments that use the same
platform. The grouping is pretty helpful, however I think it would be helpful to
have at least one demo of a feature or code in class to help the class get
started. For example, showing a simple instruction as to how to grab an object
in a game in Unity. In addition, implementing a simple VR setup such as showing
the player in game would be a helpful assignment to have and keep everyone at
the same phase and prevent them from being discourage to develop in Unity as I
certainly found it overwhelming at first if I had not known anything about Unity
or Blender experience.\par 

The collaboration groups are helpful as it allowed us to reflect on our
applications and evulate VR design factors that would help our applications work
effectively. Thus, collaboration reveal weaknesses and strengths in our
applications.  
\newpage

\section{Classmates' VR App Evaluation}
\subsection{Arjun Vijay- Golf Mania}
Testing the game I was able to move around smoothly. However, one of the
problems I was having in the game was the lack of tutorial, therefore I was not
aware on how grab an object. The game was, however, very disorienting and the
camera moved everytime the user teleports to a location therefore override the
camera orientation. \subsection{Mike - VR Fishing}
The motion in the game and frame rate was running smoothly. The user had full
control of the device in 6 degrees of freedom. Screen space reflection problem
\par

One of the main issues present in the game however is the addition of screen
space reflection with the water. Since the reflection reflect different per eye,
it tends to cause the user to be a little bit cross eyed. However, in terms of
performance the application runs well.
\par

There are no present accelerations in the game which make the game enjoyable as
I did not feel any discomfort. There are no effects that causes the user to lose
control of the player. \par

A snapping feature would be helpful and adding different fishes in the game would be helpful.

\subsection{Jonathan - VRcade(Incompatible)}
I was not able to test the game due to the following error: 
\begin{lstlisting}{language=CMD}
    adb: failed to install C:\Users\Jekko\Downloads\VR Final Demo Group
    Games\VRcade_game.apk: Failure [INSTALL_FAILED_OLDER_SDK: Failed parse
    during installPackageLI: /data/app/vmdl1887246848.tmp/base.apk (at Binary
    XML file line #9): Requires newer sdk version #29 (current version is #25)]
\end{lstlisting}
But looking at the most important factors in succesful VR applcication design 
However, looking at the video I can see the incl

\subsection{Giana Fiore - Golf Mania(Alternate)}
It was enjoyable to play this game however, the lack of tutorial had me figuring
out how to teleport/move for a good 30 seconds. But the game was made
beautifully and that addition of sound effects such as background music and item
placement made the game more engaging. I do like that there are different
environments and the user had full control of the game the whol time. There are
no accelerations present in the game and frame rate was running smoothly.

\end{document}
