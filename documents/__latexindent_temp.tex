\documentclass[12pt, twoside]{article}
\usepackage[a4paper,width=150mm,top=25mm,bottom=25mm,bindingoffset=6mm]{geometry}
\usepackage[colorinlistoftodos]{todonotes}
\usepackage{gensymb}
\usepackage{fancyhdr}
\usepackage{setspace}
\pagestyle{fancy}
\doublespace

\title{Into the Dungeon}
\author{Jekko Syquia}

\begin{document}
\maketitle
\titlepage

\tableofcontents
\newpage


\section{Introduction}

\emph{Into the dungeon} is a dungeon monster slayer VR game targeted at the
Oculus Quest device that is tethered a PC machine. The objective of the game is
to go through each level and try to slay all the monsters present in each level
of the game. Each level introduces new weapons, and with each new level the user
is introduced to new enemies. New enemies deal harder damage as the game
progresses. Each enemy AI is able to target the user and will shoot projectiles
at the user when the user is at a closer range to the player. Every enemy also
has their own score value that increases the users total score in the game. \par

This game main objective is to allow the users to experience streamlined games like
\emph{Doom} and other first-person shooter games in VR, thus enabling a tensed
gaming experience. Unlike other streamlined VR games however, the user has a
longer health bar than usual to allow the user to play at a more relaxed phased
in the game. If the user dies, the user is respawned at the starting position of
that level and the enemies are respawned. Users can also expect to keep a score
count in the game to show off their friends. It is also important to notice that
this game does not require the user to finish each level as they have the option
of skipping to the next level if they would like. Therefore, it is a "finish your
game at your own phase" and the score is dependent on the play time of the user.

\newpage
\section{VR Application Design}
One of the most important key design decision when designing this game was
balancing between physical engagement and motion sickness prevention. \par 

Balancing is most notable in the exclusion of weapons that require physical
contact with the enemies i.e. swords, hammers etc. This decision was important
as to much physical engagement such as swinging the sword could disorient users
by having them move and swing at the same time. This can make users
uncomfortable and also would require more space to work around in the game.\par

The grabbing system in the game has also been changed. While the user can still
interact with objects closely, interaction ray has been introduced to allow
users to point at a weapon at which it will be redirected to their hands at any
distance. This design decision prevents the need for users to physically reach
their weapon when it is dropped on the floor.\par

Gameplay wise, users have infinite amount of ammo, this decision was taken into
account to prevent users from needing to reload and prevents added complexity of
loading their weapons which would make this game less arcade feeling.\par

The map design simplicity was also reduced greatly. Instead of moving around the
map so much, the level was simplified to prevent users from getting lost in the
game.
\newpage
\section{Integration}
For the build-process, I primarily worked with C\# Unity and Blender as the tool
for modeling my enemies and level design. The weapons in the game was provided
through the asset store in Unity.\par 

The tool mainly used for handling VR interaction in the game was the inclusion
of \emph{XR Interaction Toolkit} through the package manager. This tool provided
the tools necessary to handling the VR device. For example, interaction with
objects in the game and player in game setup was handled by this tool. Mechanics
such as hand presence still had to be scripted and written individually. \par

Models in the game was built using Blender. This allowed for custom map models
in the game and designing custom enemies not found in the asset store. The
creation and baking of textures was also handled through Blender.  In turn these
were imported through Unity as an asset in which each enemy object has an enemy
script attached to them to handle their behavior.

\newpage
\section{User Experience}
\emph{Into the Dungeon} has many design considerations to allow for a relaxed
user experience. In this section we talk about the design decisions that make
this game much more enjoyable. It is important to note that the game was meant
to replicate arcade games like \emph{Doom} as mentioned earlier. However the
game is also meant as a way for users to physically more intense, but each level
is not required be completed as the user can skip through each level as they
would like.
\subsection{Navigation}
Users can expect to move in the game via teleportation, controller or physical
movement. Since the game utilizes \emph{Six degrees of Freedom (DOF)} users are
not limited to teleportation. Teleportation is introduced in the game to
mitigate the unconformability when moving with a joystick in the game.
Furthermore, if the user has the space perhaps a large room, the users can use
their physical space. While the games utilizes all three, they are not
restricted to one and can choose any option as they which.\par 

To ease the movement even further, the user has the option for \emph{Snap
Rotation} functionality. Snap Rotation allows users to flick the right analog
stick left or right to a fixed degree of 45\textdegree.\par

An important design decision when it comes to moving from one level to another
was the exclusion of doors. To make the game even easier to move around, I
allowed users to navigate to each level using the UI. This prevents the need to
interact with doors to go to the next level of the game.



\subsection{Guiding Audio/Sound Effects}
Every weapon and enemy interaction is present in the game. For example, the
handgun has handgun sound effects for firing to indicate it is firing.
Additionally, once the shot has been fired sound effects will be present in the
location of the collision. Therefore, there is a sense of 3D spatial sound in the
game. Once the enemy is defeated the game also lets the user know the enemy is
dead by indicating an explosion sound depending on the enemy. \par

Enemies in the game have their own individual sound as well. Each sound source
will be relative to the location of the player. By doing this, we can have
spatial awareness in the audio. Therefore, enemies shot far away will have a
more muted sound, enemies shot closer will sound closer. This rule applies to
all elements in the game to further enhance the presence of enemies in the
game.\par 

While enemy sound is present, the levels also has background music that is
playing throughout the game. Furthermore, when a player finishes a level, a
sound cue is played to indicate to the player that level is done and will be
moving to the next level.

\subsection{Text Instructional}
In the starting room the user will be given a tutorial on how to move about the
room i.e. how to use the teleporation. Additionally in the beginning room the
user will be provided with instructions on how to use the weapons by initializing UI elements that indicates what everythings is. Whenever the
user enters a new room an instruction on how to use that weapon will appear.
Once the user has read and tested the item the user can then proceed to walk forward to engage the enemies which they can immedietly notice. 

\subsection{Instructional Signs/Wayfinders}
Each room contain an instruction on how to use the weapons as well as how to
move to the next room as the user is spawned in front of the instructions.

\subsection{Six Degrees of Freedom}
\emph{Six Degrees of Freedom (DOF)} is present in the game. This design decision
was an important implementation as moving with guns around the game would be a
lot harder if the user does not have full control of their weapons.
\newpage

\section{User Guideline}

\subsection{Windows Installation}
To use the game for Windows. Extract the \emph{Windows Build V 1.2.7z} to their desired location using 7zip. Assuming the user has already installed the
necessary drivers to connect their VR to a pc. Simply execute the file \emph{Into The
Dungeon VR.exe} located in the Windows build folder. There are no further installations needed as the applications is built.
\section{Conclusion}
\section{Suggestions for Further Improvements}

\appendix
\section{Lessons Learned}
\section{Feedback to the Instructor}
\section{Classmates' VR App Evaluation}

    

\end{document}
